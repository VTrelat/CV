%-----------------------------------------------------------------------------------------------------------------------------------------------%
%	The MIT License (MIT)
%
%	Copyright (c) 2019 Jan Küster
%
%	Permission is hereby granted, free of charge, to any person obtaining a copy
%	of this software and associated documentation files (the "Software"), to deal
%	in the Software without restriction, including without limitation the rights
%	to use, copy, modify, merge, publish, distribute, sublicense, and/or sell
%	copies of the Software, and to permit persons to whom the Software is
%	furnished to do so, subject to the following conditions:
%	
%	THE SOFTWARE IS PROVIDED "AS IS", WITHOUT WARRANTY OF ANY KIND, EXPRESS OR
%	IMPLIED, INCLUDING BUT NOT LIMITED TO THE WARRANTIES OF MERCHANTABILITY,
%	FITNESS FOR A PARTICULAR PURPOSE AND NONINFRINGEMENT. IN NO EVENT SHALL THE
%	AUTHORS OR COPYRIGHT HOLDERS BE LIABLE FOR ANY CLAIM, DAMAGES OR OTHER
%	LIABILITY, WHETHER IN AN ACTION OF CONTRACT, TORT OR OTHERWISE, ARISING FROM,
%	OUT OF OR IN CONNECTION WITH THE SOFTWARE OR THE USE OR OTHER DEALINGS IN
%	THE SOFTWARE.
%	
%
%-----------------------------------------------------------------------------------------------------------------------------------------------%


%============================================================================%
%
%	DOCUMENT DEFINITION
%
%============================================================================%

%we use article class because we want to fully customize the page and don't use a cv template
\documentclass[10pt,A4]{article}	


%----------------------------------------------------------------------------------------
%	ENCODING
%----------------------------------------------------------------------------------------

% we use utf8 since we want to build from any machine
\usepackage[utf8]{inputenc}		

%----------------------------------------------------------------------------------------
%	LOGIC
%----------------------------------------------------------------------------------------

% provides \isempty test
\usepackage{xstring, xifthen}

%----------------------------------------------------------------------------------------
%	FONT BASICS
%----------------------------------------------------------------------------------------

% some tex-live fonts - choose your own

%\usepackage[defaultsans]{droidsans}
%\usepackage[default]{comfortaa}
%\usepackage{cmbright}
\usepackage[default]{raleway}
%\usepackage{fetamont}
%\usepackage[default]{gillius}
%\usepackage[light,math]{iwona}
%\usepackage[thin]{roboto} 

% set font default
\renewcommand*\familydefault{\sfdefault} 	
\usepackage[T1]{fontenc}

% more font size definitions
\usepackage{moresize}

%----------------------------------------------------------------------------------------
%	FONT AWESOME ICONS
%---------------------------------------------------------------------------------------- 

% include the fontawesome icon set
\usepackage{fontawesome}

% use to vertically center content
% credits to: http://tex.stackexchange.com/questions/7219/how-to-vertically-center-two-images-next-to-each-other
\newcommand{\vcenteredinclude}[1]{\begingroup
\setbox0=\hbox{\includegraphics{#1}}%
\parbox{\wd0}{\box0}\endgroup}

% use to vertically center content
% credits to: http://tex.stackexchange.com/questions/7219/how-to-vertically-center-two-images-next-to-each-other
\newcommand*{\vcenteredhbox}[1]{\begingroup
\setbox0=\hbox{#1}\parbox{\wd0}{\box0}\endgroup}

% icon shortcut
\newcommand{\icon}[3] { 							
	\makebox(#2, #2){\textcolor{maincol}{\csname fa#1\endcsname}}
}	

% icon with text shortcut
\newcommand{\icontext}[4]{ 						
	\vcenteredhbox{\icon{#1}{#2}{#3}}  \hspace{2pt}  \parbox{0.9\mpwidth}{\textcolor{#4}{#3}}
}

% icon with website url
\newcommand{\iconhref}[5]{ 						
    \vcenteredhbox{\icon{#1}{#2}{#5}}  \hspace{2pt} \href{#4}{\textcolor{#5}{#3}}
}

% icon with email link
\newcommand{\iconemail}[5]{ 						
    \vcenteredhbox{\icon{#1}{#2}{#5}}  \hspace{2pt} \href{mailto:#4}{\textcolor{#5}{#3}}
}

%----------------------------------------------------------------------------------------
%	PAGE LAYOUT  DEFINITIONS
%----------------------------------------------------------------------------------------

% page outer frames (debug-only)
% \usepackage{showframe}		

% we use paracol to display breakable two columns
\usepackage{paracol}

% define page styles using geometry
\usepackage[a4paper]{geometry}

% remove all possible margins
\geometry{top=5mm, bottom=5mm, left=5mm, right=5mm}

\usepackage{fancyhdr}
\pagestyle{empty}

% space between header and content
% \setlength{\headheight}{0pt}

% indentation is zero
\setlength{\parindent}{0mm}

%----------------------------------------------------------------------------------------
%	TABLE /ARRAY DEFINITIONS
%---------------------------------------------------------------------------------------- 

% extended aligning of tabular cells
\usepackage{array}

% custom column right-align with fixed width
% use like p{size} but via x{size}
\newcolumntype{x}[1]{%
>{\raggedleft\hspace{0pt}}p{#1}}%


%----------------------------------------------------------------------------------------
%	GRAPHICS DEFINITIONS
%---------------------------------------------------------------------------------------- 

%for header image
\usepackage{graphicx}

% use this for floating figures
% \usepackage{wrapfig}
% \usepackage{float}
% \floatstyle{boxed} 
% \restylefloat{figure}

%for drawing graphics		
\usepackage{tikz}				
\usetikzlibrary{shapes, backgrounds,mindmap, trees}

%----------------------------------------------------------------------------------------
%	Color DEFINITIONS
%---------------------------------------------------------------------------------------- 
\usepackage{transparent}
\usepackage{color}

% primary color
\definecolor{maincol}{RGB}{140, 108, 63}
\definecolor{emphcol}{RGB}{66, 54, 37}

% accent color, secondary
% \definecolor{accentcol}{RGB}{ 250, 150, 10 }

% dark color
\definecolor{darkcol}{RGB}{ 70, 70, 70 }

% light color
\definecolor{lightcol}{RGB}{245,245,245}


% Package for links, must be the last package used
\usepackage[hidelinks]{hyperref}

% returns minipage width minus two times \fboxsep
% to keep padding included in width calculations
% can also be used for other boxes / environments
\newcommand{\mpwidth}{\linewidth-\fboxsep-\fboxsep}
	


%============================================================================%
%
%	CV COMMANDS
%
%============================================================================%

%----------------------------------------------------------------------------------------
%	 CV LIST
%----------------------------------------------------------------------------------------

% renders a standard latex list but abstracts away the environment definition (begin/end)
\newcommand{\cvlist}[1] {
	\begin{itemize}{#1}\end{itemize}
}

%----------------------------------------------------------------------------------------
%	 CV TEXT
%----------------------------------------------------------------------------------------

% base class to wrap any text based stuff here. Renders like a paragraph.
% Allows complex commands to be passed, too.
% param 1: *any
\newcommand{\cvtext}[1] {
	\begin{tabular*}{1\mpwidth}{p{0.98\mpwidth}}
		\parbox{1\mpwidth}{#1}
	\end{tabular*}
}

%----------------------------------------------------------------------------------------
%	CV SECTION
%----------------------------------------------------------------------------------------

% Renders a a CV section headline with a nice underline in main color.
% param 1: section title
\newcommand{\cvsection}[1] {
	\vspace{5pt}
	\cvtext{
		\textbf{\LARGE{\textcolor{darkcol}{\uppercase{#1}}}}\\[-4pt]
		\textcolor{maincol}{ \rule{0.1\textwidth}{2pt} } \\
	}
}

%----------------------------------------------------------------------------------------
%	META SKILL
%----------------------------------------------------------------------------------------

% Renders a progress-bar to indicate a certain skill in percent.
% param 1: name of the skill / tech / etc.
% param 2: level (for example in years)
% param 3: percent, values range from 0 to 1
\newcommand{\cvskill}[3] {
	\begin{tabular*}{1\mpwidth}{p{0.72\mpwidth}  r}
 		\textcolor{black}{\textbf{#1}} & \textcolor{maincol}{#2}\\
	\end{tabular*}%
	
	\hspace{4pt}
	\begin{tikzpicture}[scale=1,rounded corners=2pt,very thin]
		\fill [lightcol] (0,0) rectangle (1\mpwidth, 0.15);
		\fill [maincol] (0,0) rectangle (#3\mpwidth, 0.15);
  	\end{tikzpicture}%
}


%----------------------------------------------------------------------------------------
%	 CV EVENT
%----------------------------------------------------------------------------------------

% Renders a table and a paragraph (cvtext) wrapped in a parbox (to ensure minimum content
% is glued together when a pagebreak appears).
% Additional Information can be passed in text or list form (or other environments).
% the work you did
% param 1: time-frame i.e. Sep 14 - Jan 15 etc.
% param 2:	 event name (job position etc.)
% param 3: Customer, Employer, Industry
% param 4: Short description
% param 5: work done (optional)
% param 6: technologies include (optional)
% param 7: achievements (optional)
\newcommand{\cvevent}[7] {
	
	% we wrap this part in a parbox, so title and description are not separated on a pagebreak
	% if you need more control on page breaks, remove the parbox
	\parbox{\mpwidth}{
		\begin{tabular*}{1\mpwidth}{p{0.72\mpwidth}  r}
	 		\textcolor{black}{\textbf{#2}} & \colorbox{maincol}{\makebox[0.25\mpwidth]{\textcolor{white}{#1}}} \\
			\textcolor{maincol}{\textbf{#3}} & \\
		\end{tabular*}\\[8pt]
	
		\ifthenelse{\isempty{#4}}{\vspace{-20pt}}{
			\cvtext{#4}\\
		}
	}

	\ifthenelse{\isempty{#5}}{}{
		\vspace{9pt}
		{#5}
	}

	\ifthenelse{\isempty{#6}}{}{
		\vspace{9pt}
		\cvtext{\textbf{Technologies include:}}\\
		{#6}
	}

	\ifthenelse{\isempty{#7}}{}{
		\vspace{9pt}
		\cvtext{\textbf{Achievements include:}}\\
		{#7}
	}
	\vspace{14pt}
}

%----------------------------------------------------------------------------------------
%	 CV META EVENT
%----------------------------------------------------------------------------------------

% Renders a CV event on the sidebar
% param 1: title
% param 2: subtitle (optional)
% param 3: customer, employer, etc,. (optional)
% param 4: info text (optional)
\newcommand{\cvmetaevent}[4] {
	\textcolor{maincol} {\cvtext{\textbf{\begin{flushleft}#1\end{flushleft}}}}

	\ifthenelse{\isempty{#2}}{}{
	\textcolor{darkcol} {\cvtext{\textbf{#2}} }
	}

	\ifthenelse{\isempty{#3}}{}{
		\cvtext{{ \textcolor{darkcol} {#3} }}\\
	}

	\cvtext{#4}\\[14pt]
}

%---------------------------------------------------------------------------------------
%	QR CODE
%----------------------------------------------------------------------------------------

% Renders a qrcode image (centered, relative to the parentwidth)
% param 1: percent width, from 0 to 1
\newcommand{\cvqrcode}[1] {
	\begin{center}
		\includegraphics[width={#1}\mpwidth]{qrcode}
	\end{center}
}


%============================================================================%
%
%
%
%	DOCUMENT CONTENT
%
%
%
%============================================================================%
\begin{document}
\columnratio{0.3}
\setlength{\columnsep}{2.2em}
\setlength{\columnseprule}{4pt}
\colseprulecolor{lightcol}
\begin{paracol}{2}
\begin{leftcolumn}
%---------------------------------------------------------------------------------------
%	META IMAGE
%----------------------------------------------------------------------------------------
\begin{center}
\includegraphics[width=.95\linewidth]{photocv.jpg}	%trimming relative to image size
\end{center}

%---------------------------------------------------------------------------------------
%	META SKILLS
%----------------------------------------------------------------------------------------
\cvsection{PROGRAMMATION}

\cvskill{Python, C, \LaTeX, Git} {} {1} \\[-5pt]

\cvskill{OCaml, Isabelle(HOL)} {} {.8} \\[-5pt]

\cvskill{JavaScript, TypeScript} {} {0.65} \\[-5pt]

\cvskill{React, HTML, CSS} {} {.6} \\[-5pt]

\cvskill{Go, Assembly} {} {.55} \\[-5pt]

\cvskill{B (Atelier B)} {} {.4} \\[-5pt]

\cvsection{LANGUES}

\cvskill{Français} {\hspace{11pt}Natif} {1} \\[-5pt]

\cvskill{Anglais} {\hspace{-5pt}IELTS C1} {.96} \\[-5pt]

\cvskill{Allemand} {\hspace{-45pt}Goethe Zertif. B2} {.65} \\[-5pt]

\vspace{-10pt}
\vfill\null
\cvsection{CONTACT}

\icontext{MobilePhone}{12}{+33 7 68 20 72 01}{black}\\[4pt]
\iconemail{Envelope}{12}{vincent.trelat@depinfonancy.net}{vincent.trelat@depinfonancy.net}{black}\\[4pt]
% \icontext{MapMarker}{12}{38 rue Gabriel Mouilleron\\54000 Nancy}{black}\\[4pt]

\vfill\null
\cvqrcode{0.8}

%---------------------------------------------------------------------------------------
%	EDUCATION
%----------------------------------------------------------------------------------------
%\newpage
%\cvsection{EDUCATION}
%
%\cvmetaevent
%{2009 - 2011}
%{M. Sc. Physics.}
%{Universität Bonn}
%{Main thematic priority of those master studies was numerical time series analysis of non-linear dynamical systems. Besides data analysis and transformation, great importance was attached to fast algorithms and efficient software architecture.
%
%In the master thesis a numerical approach for the detection of the direction of interaction was proposed. Analysis of this new approach was performed with the help computer simulations to find out its limits and to compare it to another commonly used approaches.
%
%This numerical approach was highly optimised for cluster computing and implemented in c++ . For those purposes a distributed computing cluster had to be set up and administrated.}
%
%\cvmetaevent
%{2006 - 2009}
%{B. Sc. Physics.}
%{Universität Bonn}
%{The topic for the bachelor's thesis was 'Feshbach resonance'. A numerical application was built to calculate the diagrams.}
%
%\vfill\null
%\cvqrcode{0.7}

%---------------------------------------------------------------------------------------
%	CERTIFICATION
%----------------------------------------------------------------------------------------
%\newpage
%\cvsection{CERTIFICATIONS}
%
%\cvmetaevent
%{LPIC 1 - Linux administrator}
%{}
%{}
%{Certificate issued by the Linux Professional Institute to prove abilities in Linux administration}
%
%\cvmetaevent
%{IBM InfoSphere Advanced DataStage Essentials}
%{}
%{}
%{Intense course about the ETL technologies and the use of IBM DataStage.}
%
%\cvmetaevent
%{Jump Start Program}
%{}
%{}
%{Two months full-time training in object oriented programming in Java SE/EE, software development, testing and modern enterprise web-frameworks. Other topics were object oriented design patterns, test-driven development, SQL-databases and webservers in Java environment (Tomcat / Glasfish / JBoss / Jety)}
%
%
%\cvmetaevent
%{Online Classes}
%{}
%{}
%{It is important for me to stay up to date with the newest topics in the field of IT. In DevOps it is also important to have a general overview and a hands-on experience on them. Therefore, besides intense article studies, I also keep myself up to date with online classes.}
%
%\vfill
%\cvqrcode{0.7}

%\newpage
%\mbox{} % hotfix to place qrcode on the bottom when there are not other elements
%\vfill
%\cvqrcode{0.7}

\end{leftcolumn}
\begin{rightcolumn}
%---------------------------------------------------------------------------------------
%	TITLE  HEADER
%----------------------------------------------------------------------------------------
\fcolorbox{white}{darkcol}{
\begin{minipage}[c][2.8cm][c]{1\mpwidth}
	\begin{center}
		\large{ \textbf{ \textcolor{white}{ \uppercase{ Vincent Trélat } } } } \\[-5pt]
		\textcolor{white}{ \rule{0.2\textwidth}{1.25pt} } \\[0pt]
		\Large{ \textcolor{white} {Étudiant Ingénieur spécialisé en\\Méthodes Formelles et leurs Applications}}
	\end {center}
\end{minipage}} \\[0pt]
\vspace{-10pt}


%---------------------------------------------------------------------------------------
%	WORK EXPERIENCE
%----------------------------------------------------------------------------------------
\vfill\null
\cvsection{FORMATION}
\vspace{-10pt}
\cvevent
	{2020 - 2023}
	{École Nationale Supérieure des Mines de Nancy}
	{Étudiant Ingénieur en Informatique, Nancy, France}
	{}
	{\cvlist{
		\item \textbf{\color{emphcol}Département Informatique} : Software Engineering, Foundation of Computing, Programming Languages, Cyber-awareness, Data Analysis, Deep Learning
		\item Événement \textbf{\color{emphcol}Advent of Code 2021} : 1er sur un classement local à l'École des Mines. Meilleur classement mondial : 200 sur plus de 237 000 participants
		\item Spécialisation en méthodes formelles et leurs applications
	}}
	{}
	{}
\vspace{-10pt}
\cvevent
	{2018 - 2020}
	{CPGE Scientifique}
	{Lycée Pothier, Orléans, France}
	{}
	{\cvlist{
		\item MPSI puis MP$^\star$, option Informatique
	}}
	{}
	{}
\vspace{-10pt}
\cvevent
	{2015 - 2018}
	{Lycée, filière scientifique}
	{Lycée Charles Péguy, Orléans, France}
	{}
	{\cvlist{
		\item Baccalauréat Scientifique Européen Anglais : mention Très Bien avec félicitations du jury
	}}
	{}
	{}

%---------------------------------------------------------------------------------------
%	WORK EXPERIENCE
%----------------------------------------------------------------------------------------
\vspace{-20pt}
\vfill\null
\cvsection{EXPÉRIENCE PROFESSIONNELLE}

\cvevent
	{05/2022 - 09/2022}
	{Clearsy, Aix-en-Provence, France}
	{Stage ingénieur R\&D Méthodes Formelles}
	{Justification formelle de la sécurité des aspects exécution en temps réel de la Clearsy Safety Platform (CSP) avec la Méthode B.}
	{}
	{}
	{}
\cvevent
	{09/2021 - 08/2022}
	{Loria, Nancy, France}
	{Stage de Recherche en Méthodes Formelles}
	{Travail sur la \emph{``Vérification formelle en Isabelle(HOL) d'un algorithme calculant les composantes fortement connexes d'un graphe''} qui a donné lieu à une {\color{emphcol}\bf publication} dans l'\textbf{\color{emphcol}Archive of Formal Proofs} :\\\href{https://www.isa-afp.org/entries/SCC_Bloemen_Sequential.html}{{\color{emphcol}\bf https://www.isa-afp.org/entries/SCC\_Bloemen\_Sequential.html}}}
	{}
	{}
	{}
\cvevent
	{06/2021 - 07/2021}
	{Valve Précision, Saint-Michel-sur-Orge, France}
	{Stage Opérateur}
	{}
	{}
	{}
	{}
\cvevent
	{Occasionnel}
	{Professeur particulier}
	{Mathématiques, physique et informatique, plus de dix élèves entre 3e et bac+2}
	{}
	{}
	{}
	{}

%---------------------------------------------------------------------------------------
%	PROFILE
%----------------------------------------------------------------------------------------
% \vspace{-10pt}
\vfill\null
\cvsection{INFORMATIONS PERSONNELLES}

% \vspace{-10pt}
\cvtext{
\begin{itemize}
	\item Photographie : réalisation de portraits et photos de produits pour présentation sur les réseaux sociaux ou sites web, pratique également en loisir.
	\item Excellente maîtrise de Photoshop, Lightroom, Final Cut Pro
	\item Musique : ancien trompettiste et guitariste autodidacte.
\end{itemize}
	
}
	
	
% hotfixes to create fake-space to ensure the whole height is used
\mbox{}
\vfill
\mbox{}
\vfill
\mbox{}
\vfill
\mbox{}
\end{rightcolumn}
\end{paracol}
\end{document}

